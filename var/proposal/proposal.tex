\documentclass[reprint,preprintnumbers,showkeys,prb,citeautoscript]{revtex4-1}
\usepackage{physics}
\usepackage{siunitx}


\newcommand\tfm{{\scshape tfm}}
\newcommand\tfpy{TFpy}

\begin{document}
\preprint{\tfpy-v0.1.0$\alpha$-PREPRINT}
\preprint{REV-1 (not for distribution)}

\title{\tfpy: A Python package to analyse and fit time-domain experimental data to transfer function model}
\author{M. Sotgia}
\email{s4942225@studenti.unige.it}
\affiliation{Dipartimento di Fisica, Università degli Studi di Genova, 16146 Genova, Italy}
\date{\today}
\keywords{bode diagrams, complex-space model, python package, transfer function}
\begin{abstract}
\tfpy\ is a Python package whose aim is to facilitate acquisition, processing and visualisation for frequency-domain data for transfer function models. 
\end{abstract}

\maketitle

\section{Introduction}

At the basis of any time-domain physical experiment --- such as Gravitational-wave detection --- is often needed to analyse frequency-domain data, usually presented in the form of a transfer function $H(\nu)$ for which the phase and amplitude diagrams --- the Bode diagrams --- are useful. Numerous Python packages already exist, most of them of broader spectrum of use \cite{2020SciPy-NMeth,pythoncontrol}, some built to meet very specific use-cases \cite{gwpy,kontrol}. All those package however refer to Fourier Transform, or Fast Fourier Transform ({\scshape fft}) to perform frequency-series calculation, and most of them is not used to fit actual experimental data to a \tfm, in a way familiar to us. \\ The \tfpy{} package is build to address this very purpose, providing a complete toolset for frequency-domain \tfm{} fit and visualisation using Bode diagrams. 

A non-tipical application for the Bode diagrams is to fit the transfer function models (\tfm) to the experimental data. The main problem to adress is the complex-valued model shape for the transfer function $H(\nu)$. Using the phasor notation the frequency $\nu$ is expressed as $s=2\pi\nu\cdot i$, where $i$ is the complex imaginary unit. This way the gain and phase are implicitly obtained from the transfer function, which in itself can be represented in the complex value notation \[H(s) = \abs{H(s)}\exp[i\phi(s)].\]


\section*{acknowledgement}
This work is possible only thanks to Francesco Polleri who gave me useful insights and to which I am in great 

% \nocite{kontrol}
\bibliography{references}

\end{document}