\documentclass[a4paper,reprint,preprintnumbers,showkeys]{revtex4-2}
\usepackage[a4paper,margin=0.6in]{geometry}
\usepackage{physics}
\usepackage{siunitx}

\newcommand\tfm{{\scshape tfm}}
\newcommand\tfpy{TFpy}

\begin{document}
\title{\tfpy: A Python package to analyse and fit time-domain experimental data to transfer function model}
\author{M. Sotgia}
\email{s4942225@studenti.unige.it}
\affiliation{Dipartimento di Fisica, Università degli Studi di Genova, I-16146 Genova, Italy}
\date{\today}
\preprint{\tfpy-PREPRINT}
\preprint{REV-1 (not for distribution)}
\keywords{transfer function; complex space model; python package}
\maketitle

\section{Introduction}

At the basis of any time-domain physical experiment --such as Gravitational-wave detection -- is often needed to analyse frequency-domain data, usually presented in the form of a transfer function $H(\nu)$ for which the phase and amplitude diagrams --the Bode diagrams-- are useful. Numerous Python packages already exist, most of them of broader spectrum of use \cite[]{2020SciPy-NMeth,pythoncontrol}, some built to meet very specific use-cases \cite[]{gwpy,kontrol}. All those package refer to Fourier Transform, or Fast Fourier Transform ({\scshape FFT}) to perform frequency-series calculation, and most of them is non used to fit actual experimental data to \tfm, in a sense familiar to us. 

A non-tipical application for the Bode diagrams is to fit the transfer function models (\tfm) to the experimental data. The main problem to adress is the complex-valued model shape for the transfer function $H(\nu)$. Using the phasor notation the frequency $\nu$ is expressed as $s=2\pi\nu\cdot i$, where $i$ is the complex imaginary unit. This way the gain and phase are implicitly obtained from the transfer function, which in itself can be represented in the complex value notation \[H(s) = \abs{H(s)}\exp[i\phi(s)].\]

% \nocite{kontrol}
\bibliography{references}

\end{document}